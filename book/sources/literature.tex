% !TEX root = ../script.tex
\section{Что еще читать про Байесовскую математическую статистику}

Существует множество хороших книг, в которых описано современное состояние Байесовской статистики и Байесовского машинного обучения.

В машинном обучении стоит начать с книги К.Бишопа~\cite{bishop2006pattern}.
Большую часть других вопросов покрывает более современная книга А.Гельмана~\cite{gelman2014bayesian}. 
Следует использовать последнее третье издание, в которое включен, например, раздел, посвященный процессам Дирихле.
С другой стороны следует отметить, что эту книгу следует использовать скорее как справочник, чем как книгу, которую следует читать подряд.

В математической статистике данное пособие больше всего коррелирует с книгой~\cite{robert2007bayesian}, богатой на примеры использования Байесовской математической статистики и ее апологии.

Наиболее полно непараметрическая Байесовская статистика изложена в книге Дж.Гоша~\cite{ghosh2007bayesian}. 
В этой книге довольного много опечаток, а изложение не всегда ясное и последовательное.
Однако она наиболее полно представляет широкое многообразие результатов как в параметрической, так и в непараметрической Байесовской статистике.


