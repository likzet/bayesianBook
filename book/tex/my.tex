\newcommand{\Sample}{\mathrm{D}} % sample
\newcommand{\sS}{n} % sample size
\newcommand{\iD}{d} % input dimension
\newcommand{\pD}{p} % parameter space dimension
\newcommand{\fN}{g} % number of factors for Variational inference

\renewcommand{\X}{\mathcal{X}} % some subset of for example \mathbb{R}^1

\newcommand{\const}{\mathrm{const}}

\newcommand{\estTk}{\tilde{\theta}_k} % an estimate of theta
% \newcommand{\mleT}{\hat{\theta}_{\mathrm{MLE}}} % MLE estimate of theta
\newcommand{\mleT}{\hat{\theta}} % MLE estimate of theta

\newcommand{\vecB}{\mathbf{b}} % b vector
\newcommand{\vecE}{\mathbf{e}} % e vector
\newcommand{\vecK}{\mathbf{k}} % k vector
\newcommand{\vecP}{\mathbf{p}} % p vector
\newcommand{\vecU}{\mathbf{u}} % u vector
\newcommand{\vecW}{\mathbf{w}} % w vector
\newcommand{\vecX}{\mathbf{x}} % x vector
\newcommand{\vecY}{\mathbf{y}} % y vector
\newcommand{\vecZ}{\mathbf{z}} % z vector

\newcommand{\vecL}{\boldsymbol{\lambda}} % theta vector
\newcommand{\vecT}{\boldsymbol{\theta}} % theta vector
\newcommand{\vecO}{\boldsymbol{\omega}} % omega vector
\newcommand{\vecMu}{\boldsymbol{\mu}} % mu vector
\newcommand{\vecEta}{\boldsymbol{\eta}} % eta vector
\newcommand{\vecTau}{\boldsymbol{\tau}} % eta vector
\newcommand{\vecA}{\boldsymbol{\alpha}} % alpha vector

\newcommand{\decSpace}{\mathbb{A}} % decision space A

\newcommand{\bbR}{\mathbb{R}} % space R
\newcommand{\bbX}{\mathbb{X}} % space X
\newcommand{\bbZ}{\mathbb{Z}} % space Z
\newcommand{\bbE}{\mathbb{E}} % expectation E
\newcommand{\bbV}{\mathbb{V}} % variance E


\newcommand{\hRatio}{m} % ratio between grid sizes for high and low fidelity data
\newcommand{\sRatio}{\delta} % ratio between grid sizes for high and low fidelity data

\newcommand{\mN}{\mathcal{N}} % Normal distribution

\newcommand{\tra}{\mathrm{T}} % transose sign

% \newcommand{\mleT}{\tilde{\vecT}} % maximum likelihood estimation for vector of parameters theta
% \newcommand{\E}{\mathbb{E}} % expectation
\newcommand{\KuLi}{\mathrm{KL}} % Kullback-Lieber divergence

\hyphenation{Ме-тро-по-ли-са-Ха-стинг-са}

% \theoremstyle[plain]
\newtheorem{Theorem}{Теорема}
\newtheorem{Lemma}{Лемма}
\newtheorem{Proposition}{Предложение}
\newtheorem{Statement}{Утверждение}
\newtheorem{Corollary}{Следствие}
\newtheorem{Remark}{Замечание}

% \theoremstyle[definition]
\newtheorem{Definition}{Определение}

\newcounter{problemNumber}
\newcommand{\problem}{\vspace{12px} \noindent\stepcounter{problemNumber}{\bf{\arabic{problemNumber}. \,\,}}}


\theoremstyle{definition}
\newtheorem{example}{Пример}[section]